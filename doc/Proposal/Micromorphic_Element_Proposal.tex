%\documentclass[11pt]{article}
\documentclass{asme2ej}
\usepackage{amsmath,amssymb,graphicx,bm}
\usepackage{listings, color, subcaption, placeins}
\usepackage{undertilde}
\usepackage{algorithm,algpseudocode}
\usepackage{multicol}
\usepackage{makecell}
\graphicspath{{./}}
%%%%%%%%%%%%%%%%%%%%%%%%%%%%%%
%%%%%%%%%%%%%%%%%%%%%%%%%%%%%%
%%%%%%%%%%%%%%%%%%%%%%%%%%%%%%

%% If you want to define a new command, you can do it like this:
\newcommand{\Q}{\mathbb{Q}}
\newcommand{\R}{\mathbb{R}}
\newcommand{\Z}{\mathbb{Z}}
\newcommand{\C}{\mathbb{C}}
\newcommand{\e}{\bm{e}}
\newcommand{\TEN}[1]{\underline{\underline{#1}}}
\newcommand{\VEC}[1]{\utilde{#1}}
\newcommand{\UVEC}[1]{\underline{#1}}
\newcommand{\PK}[1]{\TEN{\tau}^{(#1)}}
\newcommand{\cauchy}{\TEN{\sigma}}
\newcommand{\st}{$^{\text{st}}$}
\newcommand{\nd}{$^{\text{nd}}$}
\newcommand\defeq{\mathrel{\stackrel{\makebox[0pt]{\mbox{\normalfont\tiny def}}}{=}}}

\graphicspath{{./images/}}

%% If you want to use a function like ''sin'' or ''cos'', you can do it like this
%% (we probably won't have much use for this)
% \DeclareMathOperator{\sin}{sin}   %% just an example (it's already defined)


\begin{document}
\title{Micromorphic Finite Element Code}
\author{Nathan Miller}

\maketitle

\begin{abstract}

A proposal for the development of a finite element framework within which the micromorphic equations of motion can be solved is detailed. The equations are presented along with some background on their derivation and meaning. The equations are to be written in such a way that an Abaqus, ``UEL,'' or a, ``user element,'' can be relatively easily developed. The code will be developed in Python which allows rapid iteration. The overall development strategy is also detailed.
\end{abstract}

\section{Nomenclature}

\begin{table}[htb!]
\centering
\begin{tabular}{|c|l|}
\hline
$x_i$ & The position of the center of mass of the differential element in the\\
& current configuration\\
\hline
$X_I$ & The position of the center of mass of the differential element in the\\
& reference configuration\\
\hline
$\xi_I$ & The position of the center of mass of the micro element in the\\
& current configuration w.r.t. the center of mass of the differential element\\
\hline
$\Xi_I$ & The position of the center of mass of the micro element in the\\
& reference configuration w.r.t. the center of mass of the differential element\\
\hline
$\sigma_{ij}$ & The unsymmetric Cauchy stress\\
\hline
$s_{ij}$ & The symmetric micro stress\\
\hline
$m_{ijk}$ & The higher order couple stress\\
\hline
$f_{i}$ & The body force density\\
\hline
$a_{i}$ & The acceleration\\
\hline
$\l_{ij}$ & The body force couple\\
\hline
$\omega_{ij}$ & The micro-spin inertia\\
\hline
\end{tabular}
\end{table}

\FloatBarrier
%
\section{Introduction}

Finite element simulations of heterogeneous materials constructed using classical finite elements frequently have difficulty capturing the internal stress-strain behavior which can be a primary driving force in the overall material response. Of particular interest are the responses of highly heterogeneous materials such as polymer bonded crystalline materials which are comprised of a hard crystalline component bound in a polymeric matrix. These materials can be found in formulations as varied as asphalt or plastic explosives.

Several approaches have been developed to address these concerns but few approach it with the generality of Eringen and Suhubi(~\cite{bib:eringen64} and \cite{bib:eringen64_2}) in their, theory of ``micromorphic,'' continuum mechanics. This approach holds the promise of both representing the effects of the microscale on macroscale response through a straightforward coupling as well as the ability to develop constitutive models which can capture this response directly. The utilization of this framework however requires specialized finite elements which are capable of handing the multi-field nature.
%
\section{Technical Approach}

\subsection{Theoretical Background}

The full derivation of the micromorphic constitutive equations is very detailed with full derivations being presented in Regueiro~\cite{bib:regueiro_micro10} and the theory manual for the code~\cite{bib:miller17}. Briefly, the equations of motion which must be solved are the balance of linear momentum

\begin{equation}
\sigma_{ji,j} + \rho \left(f_i - a_i\right) = 0\\
\end{equation}

and the balance of the, ``first moment of momentum''

\begin{equation}
\sigma_{ij} - s_{ij} + m_{kji,k} + \rho \left(l_{ji} - \omega_{ji}\right) = 0\\
\end{equation}

where the $\left(\cdot\right)_{,j}$ indicates the derivative with respect to $x$ in the current coordinate system. We note that the terms in the balance equations are volume and area averages of quantities defined in the micro scale (indicated by $\left(\cdot\right)'$ via
\begin{align*}
\rho dv &\defeq \int_{dv} \rho' dv'\\
\sigma_{ji}n_j da &\defeq \int_{da} \sigma_{ji}'n_j'da'\\
\rho f_i dv &\defeq \int_{dv} \rho' f_i' dv'\\
\rho a_i dv &\defeq \int_{dv} \rho' a_i' dv'\\
s_{ij} dv &\defeq \int_{dv} \sigma_{ij}' dv'\\
m_{ijm} n_i da &\defeq \int_{da} \sigma_{ij}' \xi_m n_i' da'\\
\rho l_{ij} dv &\defeq \int_{dv} \rho' f_i' \xi_j dv'\\
\rho \omega_{ij} dv &\defeq \int_{dv} \rho' \ddot{\xi}_i \xi_j dv'\\
\end{align*}

We define a mapping between the current and reference configurations for the position of $dv$ and $dv'$ as
\begin{align*}
F_{iI} &\defeq \frac{\partial x_i}{\partial X_I}\\
\xi_i  &\defeq \chi_{iI}\Xi_I = \left(\delta_{iI}+\phi_{iI}\right)\Xi_I\\
\end{align*}

where we note the difference that $F_{iI}$ maps $dX_I$ into $dx_i$ through the differential relationship whereas $\chi_{iI}$ is purely a linear map between the configurations and is not defined through the differential elements.

Conceptually then, we can understand the balance of first moment of momentum in the absence of body couples and micro-spin as the statement that the total stress of the body is the volume average of all the micro stresses ($s_{ij}$) added to the couple produced by the micro stresses acting on the lever arm $\xi_i$. The body couple results from a heterogeneous distribution of the body force per unit density and the micro-spin inertia results from the acceleration of the micro position vectors.

\subsection{Algorithms}

The equations of motion will be solved using the finite element method in a so-called, ``Total Lagrangian,'' configuration. We do this by mapping the stresses back to the reference configuration (for details see Regueiro~\cite{bib:regueiro_micro10} or the theory manual~\cite{bib:miller17}) to find the balance of linear momentum for a single element $e$

\begin{align*}
\sum_{n=1}^{N^{nodes,e}} c^{n,e}_j \bigg\{&\int_{\partial \hat{\mathcal{B}}^{0,t,e}} \hat{N}^{n,e} F_{jJ} S_{IJ} \hat{J} \left(\frac{\partial X_{I}}{\partial \xi_{\hat{i}}}\right)^{-1} \hat{N}_{\hat{i}} d\hat{A}& + \int_{\hat{\mathcal{B}}^{0,e}} \big\{- \hat{N}^{n,e}_{,I} S_{IJ} F_{jJ} + \hat{N}^{n,e} \rho^0 \left(f_j - a_j\right) \big\} \hat{J} d\hat{V} = \mathcal{F}_j^{n,e}\bigg\}\\
\end{align*}

where we have transformed the equations into the element coordinate system $\xi$ (indicated by $\hat{\left(\cdot\right)}$), $N$ is the shape function, $S_{IJ}$ is the second Piola Kirchhoff stress, and $\mathcal{F}_j^{n,e}$ is the residual.

We also write the balance of the first moment of momentum as
\begin{align*}
\sum_{n=1}^{N^{nodes,e}} \eta_{ij}^{n,e} &\bigg\{\int_{\mathcal{B}^{0,e}}  \bigg\{\hat{N}^{n,e} \left(F_{iI} \left(S_{IJ}-\Sigma_{IJ}\right) F_{jJ} + \rho^0\left(l_{ji} - \omega_{ji} \right)\right)  - \frac{\partial \hat{N}^{n,e}}{\partial \xi_{\hat{i}}} \left(\frac{\partial X_{K}}{\partial \xi_{\hat{i}}}\right)^{-1} F_{jJ} \chi_{iI}  M_{KJI} \bigg\} \hat{J} d\hat{V}\\
& + \int_{\partial \mathcal{B}^{0,t,e}} F_{jJ} \chi_{iI}  M_{KJI} \hat{N}^n \hat{J} \left(\frac{\partial X_{K}}{\partial \xi_{\hat{i}}}\right)^{-1} \hat{N}_{\hat{i}} d\hat{A} = \mathcal{M}_{ij}^{n,e} \bigg\}\\
\end{align*}

We will organize these residuals into the residual vector using the following approach for a linear 8 noded hex element
\begin{align*}
\mathcal{R}^e &= \left\{\begin{array}{c}
\mathcal{F}_j^{1,e}\\
\mathcal{M}_j^{1,e}\\
\mathcal{F}_j^{2,e}\\
\mathcal{M}_j^{2,e}\\
\vdots\\
\mathcal{F}_j^{8,e}\\
\mathcal{M}_j^{8,e}\\
\end{array}\right\}
\end{align*}

where
\begin{align*}
\mathcal{M}_j^{n,e} = \left\{\begin{array}{c}
\mathcal{M}_{11}^{n,e}\\
\mathcal{M}_{22}^{n,e}\\
\mathcal{M}_{33}^{n,e}\\
\mathcal{M}_{23}^{n,e}\\
\mathcal{M}_{13}^{n,e}\\
\mathcal{M}_{12}^{n,e}\\
\mathcal{M}_{32}^{n,e}\\
\mathcal{M}_{31}^{n,e}\\
\mathcal{M}_{21}^{n,e}
\end{array}\right\}
\end{align*}

We solve the nonlinear equations using Newton Raphson which means we write the tangent vector as
\begin{align*}
\mathcal{J}^e &= -\left[\begin{array}{cccccc}
\frac{\partial \mathcal{F}_{1}^{1,e}}{\partial u_1^{1,e}} & \frac{\partial \mathcal{F}_{1}^{1,e}}{\partial u_2^{1,e}} & \frac{\partial \mathcal{F}_{1}^{1,e}}{\partial u_3^{1,e}} & \frac{\partial \mathcal{F}_{1}^{1,e}}{\partial \phi_{11}^{1,e}} & \cdots \frac{\partial \mathcal{F}_{1}^{1,e}}{\partial \phi_{21}^{8,e}}\\
\frac{\partial \mathcal{F}_{2}^{1,e}}{\partial u_1^{1,e}} & \frac{\partial \mathcal{F}_{2}^{1,e}}{\partial u_2^{1,e}} & \frac{\partial \mathcal{F}_{2}^{1,e}}{\partial u_3^{1,e}} & \frac{\partial \mathcal{F}_{2}^{1,e}}{\partial \phi_{11}^{1,e}} & \cdots \frac{\partial \mathcal{F}_{1}^{1,e}}{\partial \phi_{21}^{8,e}}\\
\vdots & \vdots & \vdots & \vdots & \ddots & \vdots\\
\frac{\partial \mathcal{M}_{2}^{8,e}}{\partial u_1^{1,e}} & \frac{\partial \mathcal{M}_{2}^{8,e}}{\partial u_2^{1,e}} & \frac{\partial \mathcal{M}_{2}^{8,e}}{\partial u_3^{1,e}} & \frac{\partial \mathcal{M}_{2}^{8,e}}{\partial \phi_{11}^{1,e}} & \cdots \frac{\partial \mathcal{M}_{1}^{8,e}}{\partial \phi_{21}^{8,e}}\\
\end{array}\right]
\end{align*}

where the superscript numbers indicate the node number. This is a $96 \times 96$ matrix. We then assemble the individual jacobian and residual to form the global jacobian and residual. The new estimate of the degree of freedom vector $\mathcal{U}_I^{n+1}$ is found by solving
\begin{align*}
\mathcal{U}_I^{n+1} &= \mathcal{U}_I^{n} + \left(\mathcal{J}^{e}\right)_{IJ}^{-1}\mathcal{R}_J
\end{align*}

At this time we will neglect kinetic effects unless there is enough time to implement this feature in the code.
%
\section{Software Environment}

\subsection{Language}

The language which will be utilized for the model will be Python 2.7. The language has been chosen both for its utility as well as its generality. The numpy library will be utilized for storing arrays though only one and two dimensional arrays will be utilized. This is to avoid the use of higher dimensional arrays. Fortran allows up to seven dimensions but a general solution is sought.

All other additional libraries to be used are either included, or generated as a part of the project. A convenient distribution of python which contains all of these libraries natively is Anaconda (\verb|www.continuum.io|). The distribution is available for download for free online.

\subsection{Implementation Design}

The code will be divided into four major modules with additional modules included as needed.

\begin{table}[htb!]
\centering
\begin{tabular}{|c|l|}
\hline
\verb|fea_driver.py| & A module containing the finite element driver program which will\\
& handle the construction and solution of a finite element model\\
\hline
\verb|micro_element.py| & A module containing the commands required for the micromorphic\\
& implementation of the hexehedral element. The format of the major\\
& function call has been chosen to mirror the Abaqus UEL subroutine\\
\hline
\verb|hex8.py| & A module containing utility commands as well as commands specific\\
& to the eight node hexehedral element\\
\hline
\verb|micromorphic_linear_elasticity.py| & A subroutine containing an implementation of a micromorphic linear\\
& elastic constitutive model\\
\hline
\end{tabular}
\end{table}

As mentioned previously, the implemented code will not utilize the higher dimensional array capabilities of numpy in order to maintain generality and ease of porting to Fortan in the future. This may cause the code to run significantly slower than if numpy array indexing was utilized. Since it is not desired to use the Python script for any significant calculations, this is deemed acceptable.



\subsection{Test Strategy}

The verification test strategy will be broken into two components, small unit tests and larger regression tests. At this time no validation simulations are intended as the model will need to be calibrated to unavailable data. Future efforts will pursue this aim.

\subsubsection{Unit Tests}

Unit tests are defined as tests which are performed to ensure a small functions capability in performing its required task correctly. These functions are, typically, not exercises of the primary functions of the code, but rather functions which are used in code execution. An example of a test which falls under this category would be the exercise of a indexing function which converts tensor notation to an index in a 1D array.

These verification tests are performed by using the \verb|unittest| module in python. This module allows verification tests to be performed by issuing the command \verb|>python module_name.py -v| where \verb|module_name.py| is the name of the python module. The results of these tests will be stored in \verb|\src\python\tests\unittests\*_unittests.tex| and will be added to the final \LaTeX~report.

If possible, all subroutines in every module should have at least one test problem which show that the code is performing as intended. In particular, it is important to show that the residuals and their tangents are consistent with each other. To this end, a utility module \verb|finite_difference.py| has been developed which allows the analytic derivatives to be evaluated against a numeric standard. In conjunction with purely analytic solutions this will help to verify the capabilities of the subroutines.

\subsubsection{Regression Tests}

Regression tests are defined as tests which are performed to exercise some part of the primary function of the code. These tests are intended to prove the capability of the code for solving some problem of interest and will be used to verify the accuracy of the functions of the code against analytic solutions (if one such solution is found to exist), manufactured solutions, and to be used in code-to-code verification.

As no analytical solutions for large deformation micromorphic continuum materials are known to exist, the current intent is to utilize the method of manufactured solutions~\cite{bib:alari2000} to develop solutions by propagating deformation fields through the PDE to determine the forcing function. This residual coupled with the appropriate boundary conditions will then be used to show that the code can converge to the correct solution. Several fields will be investigated including linear, quadratic, and other non-polynomial constructions.

\subsection{Version Control Plan}

Version control will be handled using Git (\verb|https://git-scm.com/|) in a local repository which will be uploaded to an online repository. At this time, it is not desired to make all of the work public though this is desired at a future date. The documentation, source code, and all utilities required for the verification and regression tests are located in the repository.
%
\section{Documentation Plan}

\subsection{Style Guide}

All functions should clearly describe what task they perform and, as often as possible, should be between 5-6 lines long to minimize confusion. Variable names should also be unambiguous as to their contents and should refrain from using more than two dimensions for vector arrays. All functions should have a documentation string which provides a description of the function and, if necessary, descriptions of the incoming and outgoing variables. If possible, the variable names should make this clear without documentation.

\subsection{Programmer's Manual}

The programmers manual will detail what is required for the development of new material model user subroutines and will be 1-2 pages in length. Further documentation will be handled through the style of clear variable and function names as well as the function documentation strings.

\subsection{User's Manual}

The users manual will detail how to define a new finite element mesh, material properties, and how to run the simulation. This manual is unlikely to be longer than 1 page.

\subsection{Theory Manual}

The theory manual will provide a complete reference to the equations being used and their implementation in the code. The theory manual is provided in the form of a \LaTeX~presentation which will provide detailed derivation information.

It is currently intended for the test information to be included in the report rather than in the theory manual. This seems to be a more natural place for the inclusion of such information.


%
\section{Delivery and Release Mechanism}

The code will be delivered via a Git repository which can be cloned into. The location of this repository is still being decided. The repository name will be \verb|Micromorphic_UEL|.
%
\section{Cost Estimates}



\FloatBarrier

\clearpage

\bibliographystyle{asme2ej}
\bibliography{micromorphic}

\end{document}