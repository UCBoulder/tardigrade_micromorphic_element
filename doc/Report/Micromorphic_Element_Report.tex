%\documentclass[11pt]{article}
\documentclass{asme2ej}
\usepackage{amsmath,amssymb,graphicx,bm}
\usepackage{listings, color, subcaption, placeins}
\usepackage{undertilde}
\usepackage{algorithm,algpseudocode}
\graphicspath{{./}}
%%%%%%%%%%%%%%%%%%%%%%%%%%%%%%
%%%%%%%%%%%%%%%%%%%%%%%%%%%%%%
%%%%%%%%%%%%%%%%%%%%%%%%%%%%%%

%% If you want to define a new command, you can do it like this:
\newcommand{\Q}{\mathbb{Q}}
\newcommand{\R}{\mathbb{R}}
\newcommand{\Z}{\mathbb{Z}}
\newcommand{\C}{\mathbb{C}}
\newcommand{\e}{\bm{e}}
\newcommand{\TEN}[1]{\underline{\underline{#1}}}
\newcommand{\VEC}[1]{\utilde{#1}}
\newcommand{\UVEC}[1]{\underline{#1}}
\newcommand{\PK}[1]{\TEN{\tau}^{(#1)}}
\newcommand{\cauchy}{\TEN{\sigma}}
\newcommand{\st}{$^{\text{st}}$}
\newcommand{\nd}{$^{\text{nd}}$}
\newcommand\defeq{\mathrel{\stackrel{\makebox[0pt]{\mbox{\normalfont\tiny def}}}{=}}}

\graphicspath{{./images/}}

%% If you want to use a function like ''sin'' or ''cos'', you can do it like this
%% (we probably won't have much use for this)
% \DeclareMathOperator{\sin}{sin}   %% just an example (it's already defined)


\begin{document}
\title{A Constitutive Model for Porous Polymer-Bonded Granular Materials}
\author{Nathan Miller}

\maketitle

\begin{abstract}
Polymer bonded granular materials are important engineering materials commonly arising in materials as diverse as asphalt and rocket propellants. These materials exhibit complex behavior which is strongly influenced by the interactions of the constituent materials. Experimentation on the polymer bonded explosives PBX~9501 and PBX~9502 reveals features of viscoelasticity, viscoplasticity, and a weakening of the elastic moduli which evolves viscously e.g. a, ``viscodamage.''

Previous modeling attempts have used a damaging parameter to keep track of this behavior. While effective, this approach involves some amount of obfuscation of the physical mechanisms at play. We seek in this paper to develop a model which captures the weakening of the moduli by the evolution of the porosity, defined as
\begin{align*}
\phi &\defeq \frac{dv^v}{dv}
\end{align*}
where $dv^v$ is the differential volume occupied by the voids and $dv$ is the total differential volume both in the current configuration.
\end{abstract}

%\section{Nomenclature}

\begin{table}[htb!]
\centering
\begin{tabular}{|c|l|}
\hline
$x_i$ & The position of the center of mass of the differential element in the\\
& current configuration\\
\hline
$X_I$ & The position of the center of mass of the differential element in the\\
& reference configuration\\
\hline
$\xi_I$ & The position of the center of mass of the micro element in the\\
& current configuration w.r.t. the center of mass of the differential element\\
\hline
$\Xi_I$ & The position of the center of mass of the micro element in the\\
& reference configuration w.r.t. the center of mass of the differential element\\
\hline
$\sigma_{ij}$ & The unsymmetric Cauchy stress\\
\hline
$s_{ij}$ & The symmetric micro stress\\
\hline
$m_{ijk}$ & The higher order couple stress\\
\hline
$f_{i}$ & The body force density\\
\hline
$a_{i}$ & The acceleration\\
\hline
$\l_{ij}$ & The body force couple\\
\hline
$\omega_{ij}$ & The micro-spin inertia\\
\hline
\end{tabular}
\end{table}

\FloatBarrier
%
%\section{Introduction}

Finite element simulations of heterogeneous materials constructed using classical finite elements frequently have difficulty capturing the internal stress-strain behavior which can be a primary driving force in the overall material response. Of particular interest are the responses of highly heterogeneous materials such as polymer bonded crystalline materials which are comprised of a hard crystalline component bound in a polymeric matrix. These materials can be found in formulations as varied as asphalt or plastic explosives.

Several approaches have been developed to address these concerns but few approach it with the generality of Eringen and Suhubi(~\cite{bib:eringen64} and \cite{bib:eringen64_2}) in their, theory of ``micromorphic,'' continuum mechanics. This approach holds the promise of both representing the effects of the microscale on macroscale response through a straightforward coupling as well as the ability to develop constitutive models which can capture this response directly. The utilization of this framework however requires specialized finite elements which are capable of handing the multi-field nature.
%
%\input{./tex_files/mpm_derivation}
%
%\input{./tex_files/verification}
%
%\input{./tex_files/discussion}
%
%\input{./tex_files/conclusion}

\input{./tex/derivation}

\input{./tex/implementation}

\FloatBarrier

\bibliographystyle{asme2ej}
\bibliography{mpm}

\end{document}