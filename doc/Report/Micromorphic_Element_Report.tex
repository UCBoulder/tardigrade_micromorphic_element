%\documentclass[11pt]{article}
\documentclass{asme2ej}
\usepackage{amsmath,amssymb,graphicx,bm}
\usepackage{listings, xcolor, subcaption, placeins}
\usepackage{undertilde}
\usepackage{algorithm,algpseudocode}
\usepackage{multicol}
\usepackage{makecell}
\usepackage[table]{colortbl}
\graphicspath{{./images}}
%%%%%%%%%%%%%%%%%%%%%%%%%%%%%%
%%%%%%%%%%%%%%%%%%%%%%%%%%%%%%
%%%%%%%%%%%%%%%%%%%%%%%%%%%%%%

%% If you want to define a new command, you can do it like this:
\newcommand{\Q}{\mathbb{Q}}
\newcommand{\R}{\mathbb{R}}
\newcommand{\Z}{\mathbb{Z}}
\newcommand{\C}{\mathbb{C}}
\newcommand{\e}{\bm{e}}
\newcommand{\TEN}[1]{\underline{\underline{#1}}}
\newcommand{\VEC}[1]{\utilde{#1}}
\newcommand{\UVEC}[1]{\underline{#1}}
\newcommand{\PK}[1]{\TEN{\tau}^{(#1)}}
\newcommand{\cauchy}{\TEN{\sigma}}
\newcommand{\st}{$^{\text{st}}$}
\newcommand{\nd}{$^{\text{nd}}$}
\newcommand\defeq{\mathrel{\stackrel{\makebox[0pt]{\mbox{\normalfont\tiny def}}}{=}}}

\graphicspath{{./images/}}

%% If you want to use a function like ''sin'' or ''cos'', you can do it like this
%% (we probably won't have much use for this)
% \DeclareMathOperator{\sin}{sin}   %% just an example (it's already defined)


\begin{document}
\title{Micromorphic Finte Element Implementation}
\author{Nathan Miller}

\maketitle

\begin{abstract}
A micromorphic finite element is developed along with the documentation for the implementation. The equations are presented along with some background on their derivation and meaning. The equations are to be written in such a way that an Abaqus, ``UEL,'' or a, ``user element,'' can be relatively easily developed. The code is currently implemented in Python which allows rapid iteration. The results of the regression and unit tests are included along with the current development goals.
\end{abstract}

\section{Nomenclature}

\begin{table}[htb!]
\centering
\begin{tabular}{|c|l|}
\hline
$x_i$ & The position of the center of mass of the differential element in the\\
& current configuration\\
\hline
$X_I$ & The position of the center of mass of the differential element in the\\
& reference configuration\\
\hline
$\xi_I$ & The position of the center of mass of the micro element in the\\
& current configuration w.r.t. the center of mass of the differential element\\
\hline
$\Xi_I$ & The position of the center of mass of the micro element in the\\
& reference configuration w.r.t. the center of mass of the differential element\\
\hline
$\sigma_{ij}$ & The unsymmetric Cauchy stress\\
\hline
$s_{ij}$ & The symmetric micro stress\\
\hline
$m_{ijk}$ & The higher order couple stress\\
\hline
$f_{i}$ & The body force density\\
\hline
$a_{i}$ & The acceleration\\
\hline
$\l_{ij}$ & The body force couple\\
\hline
$\omega_{ij}$ & The micro-spin inertia\\
\hline
\end{tabular}
\end{table}

\FloatBarrier
\section{Technical Approach}

\subsection{Theoretical Background}

The full derivation of the micromorphic constitutive equations is very detailed with full derivations being presented in Regueiro~\cite{bib:regueiro_micro10} and the theory manual for the code~\cite{bib:miller17}. Briefly, the equations of motion which must be solved are the balance of linear momentum

\begin{equation}
\sigma_{ji,j} + \rho \left(f_i - a_i\right) = 0\\
\end{equation}

and the balance of the, ``first moment of momentum''

\begin{equation}
\sigma_{ij} - s_{ij} + m_{kji,k} + \rho \left(l_{ji} - \omega_{ji}\right) = 0\\
\end{equation}

where the $\left(\cdot\right)_{,j}$ indicates the derivative with respect to $x$ in the current coordinate system. We note that the terms in the balance equations are volume and area averages of quantities defined in the micro scale (indicated by $\left(\cdot\right)'$ via
\begin{align*}
\rho dv &\defeq \int_{dv} \rho' dv'\\
\sigma_{ji}n_j da &\defeq \int_{da} \sigma_{ji}'n_j'da'\\
\rho f_i dv &\defeq \int_{dv} \rho' f_i' dv'\\
\rho a_i dv &\defeq \int_{dv} \rho' a_i' dv'\\
s_{ij} dv &\defeq \int_{dv} \sigma_{ij}' dv'\\
m_{ijm} n_i da &\defeq \int_{da} \sigma_{ij}' \xi_m n_i' da'\\
\rho l_{ij} dv &\defeq \int_{dv} \rho' f_i' \xi_j dv'\\
\rho \omega_{ij} dv &\defeq \int_{dv} \rho' \ddot{\xi}_i \xi_j dv'\\
\end{align*}

We define a mapping between the current and reference configurations for the position of $dv$ and $dv'$ as
\begin{align*}
F_{iI} &\defeq \frac{\partial x_i}{\partial X_I}\\
\xi_i  &\defeq \chi_{iI}\Xi_I = \left(\delta_{iI}+\phi_{iI}\right)\Xi_I\\
\end{align*}

where we note the difference that $F_{iI}$ maps $dX_I$ into $dx_i$ through the differential relationship whereas $\chi_{iI}$ is purely a linear map between the configurations and is not defined through the differential elements.

Conceptually then, we can understand the balance of first moment of momentum in the absence of body couples and micro-spin as the statement that the total stress of the body is the volume average of all the micro stresses ($s_{ij}$) added to the couple produced by the micro stresses acting on the lever arm $\xi_i$. The body couple results from a heterogeneous distribution of the body force per unit density and the micro-spin inertia results from the acceleration of the micro position vectors.

\subsection{Algorithms}

The equations of motion will be solved using the finite element method in a so-called, ``Total Lagrangian,'' configuration. We do this by mapping the stresses back to the reference configuration (for details see Regueiro~\cite{bib:regueiro_micro10} or the theory manual~\cite{bib:miller17}) to find the balance of linear momentum for a single element $e$

\begin{align*}
\sum_{n=1}^{N^{nodes,e}} c^{n,e}_j \bigg\{&\int_{\partial \hat{\mathcal{B}}^{0,t,e}} \hat{N}^{n,e} F_{jJ} S_{IJ} \hat{J} \left(\frac{\partial X_{I}}{\partial \xi_{\hat{i}}}\right)^{-1} \hat{N}_{\hat{i}} d\hat{A}& + \int_{\hat{\mathcal{B}}^{0,e}} \big\{- \hat{N}^{n,e}_{,I} S_{IJ} F_{jJ} + \hat{N}^{n,e} \rho^0 \left(f_j - a_j\right) \big\} \hat{J} d\hat{V} = \mathcal{F}_j^{n,e}\bigg\}\\
\end{align*}

where we have transformed the equations into the element coordinate system $\xi$ (indicated by $\hat{\left(\cdot\right)}$), $N$ is the shape function, $S_{IJ}$ is the second Piola Kirchhoff stress, and $\mathcal{F}_j^{n,e}$ is the residual.

We also write the balance of the first moment of momentum as
\begin{align*}
\sum_{n=1}^{N^{nodes,e}} \eta_{ij}^{n,e} &\bigg\{\int_{\mathcal{B}^{0,e}}  \bigg\{\hat{N}^{n,e} \left(F_{iI} \left(S_{IJ}-\Sigma_{IJ}\right) F_{jJ} + \rho^0\left(l_{ji} - \omega_{ji} \right)\right)  - \frac{\partial \hat{N}^{n,e}}{\partial \xi_{\hat{i}}} \left(\frac{\partial X_{K}}{\partial \xi_{\hat{i}}}\right)^{-1} F_{jJ} \chi_{iI}  M_{KJI} \bigg\} \hat{J} d\hat{V}\\
& + \int_{\partial \mathcal{B}^{0,t,e}} F_{jJ} \chi_{iI}  M_{KJI} \hat{N}^n \hat{J} \left(\frac{\partial X_{K}}{\partial \xi_{\hat{i}}}\right)^{-1} \hat{N}_{\hat{i}} d\hat{A} = \mathcal{M}_{ij}^{n,e} \bigg\}\\
\end{align*}

We will organize these residuals into the residual vector using the following approach for a linear 8 noded hex element
\begin{align*}
\mathcal{R}^e &= \left\{\begin{array}{c}
\mathcal{F}_j^{1,e}\\
\mathcal{M}_j^{1,e}\\
\mathcal{F}_j^{2,e}\\
\mathcal{M}_j^{2,e}\\
\vdots\\
\mathcal{F}_j^{8,e}\\
\mathcal{M}_j^{8,e}\\
\end{array}\right\}
\end{align*}

where
\begin{align*}
\mathcal{M}_j^{n,e} = \left\{\begin{array}{c}
\mathcal{M}_{11}^{n,e}\\
\mathcal{M}_{22}^{n,e}\\
\mathcal{M}_{33}^{n,e}\\
\mathcal{M}_{23}^{n,e}\\
\mathcal{M}_{13}^{n,e}\\
\mathcal{M}_{12}^{n,e}\\
\mathcal{M}_{32}^{n,e}\\
\mathcal{M}_{31}^{n,e}\\
\mathcal{M}_{21}^{n,e}
\end{array}\right\}
\end{align*}

We solve the nonlinear equations using Newton Raphson which means we write the tangent vector as
\begin{align*}
\mathcal{J}^e &= -\left[\begin{array}{cccccc}
\frac{\partial \mathcal{F}_{1}^{1,e}}{\partial u_1^{1,e}} & \frac{\partial \mathcal{F}_{1}^{1,e}}{\partial u_2^{1,e}} & \frac{\partial \mathcal{F}_{1}^{1,e}}{\partial u_3^{1,e}} & \frac{\partial \mathcal{F}_{1}^{1,e}}{\partial \phi_{11}^{1,e}} & \cdots \frac{\partial \mathcal{F}_{1}^{1,e}}{\partial \phi_{21}^{8,e}}\\
\frac{\partial \mathcal{F}_{2}^{1,e}}{\partial u_1^{1,e}} & \frac{\partial \mathcal{F}_{2}^{1,e}}{\partial u_2^{1,e}} & \frac{\partial \mathcal{F}_{2}^{1,e}}{\partial u_3^{1,e}} & \frac{\partial \mathcal{F}_{2}^{1,e}}{\partial \phi_{11}^{1,e}} & \cdots \frac{\partial \mathcal{F}_{1}^{1,e}}{\partial \phi_{21}^{8,e}}\\
\vdots & \vdots & \vdots & \vdots & \ddots & \vdots\\
\frac{\partial \mathcal{M}_{2}^{8,e}}{\partial u_1^{1,e}} & \frac{\partial \mathcal{M}_{2}^{8,e}}{\partial u_2^{1,e}} & \frac{\partial \mathcal{M}_{2}^{8,e}}{\partial u_3^{1,e}} & \frac{\partial \mathcal{M}_{2}^{8,e}}{\partial \phi_{11}^{1,e}} & \cdots \frac{\partial \mathcal{M}_{1}^{8,e}}{\partial \phi_{21}^{8,e}}\\
\end{array}\right]
\end{align*}

where the superscript numbers indicate the node number. This is a $96 \times 96$ matrix. We then assemble the individual jacobian and residual to form the global jacobian and residual. The new estimate of the degree of freedom vector $\mathcal{U}_I^{n+1}$ is found by solving
\begin{align*}
\mathcal{U}_I^{n+1} &= \mathcal{U}_I^{n} + \left(\mathcal{J}^{e}\right)_{IJ}^{-1}\mathcal{R}_J
\end{align*}

At this time we will neglect kinetic effects unless there is enough time to implement this feature in the code.

\section{Upcoming Enhancements}

\begin{enumerate}
\item Discover and remedy bug in fea\_solver.py that is preventing convergence. Suspicions are that the tangent has been implemented incorrectly.
\item Convert tensor representation in C++ code to vector-matrix representation to speed up the code.
\item Add in dynamic loading to the element.
\item Add in distributed loading.
\end{enumerate}


\input{./tex/unittest_results.tex}
%\input{./tex/regression_test_results.tex}



\FloatBarrier

\bibliographystyle{asme2ej}
\bibliography{micromorphic}

\end{document}